\section{ReactNative}
\label{sec:Frameorks_ReactNative}

Das von Meta, ehemals Facebook, entwickelte ReactNative Framework ermöglicht die Cross-Plattform Entwicklung mit Webtechnologien.
Dabei kommt die JavaScript Bibliothek React zum Einsatz, welche die Entwicklung von Web-\acp{UI} vereinfacht.
% Plattformunterstützung, Verbreitung

Obwohl ReactNative wie Cordova und Capacitor auf Webtechnologien, insbesondere JavaScript, basiert, wählt das Framework einen anderen Ansatz, um die Cross-Plattform Entwicklung zu ermöglichen.
ReactNative verzichtet auf den Einsatz einer WebView und nutzt stattdessen die nativen \ac{UI}-Komponenten der jeweiligen Plattformen.
Der JavaScript Code wird nicht, wie bei Foreign Language Apps nach Nunkesser \cite{Nunkesser_Taxonomy_Apps}, in nativ ausführbaren Code übersetzt, sondern in einer JavaScript-Runtime außerhalb eines Browsers ausgeführt.
Somit lässt sich ReactNative weder den Hybrid Web Apps noch den Foreign Language Apps zuordnen.
%TODO: umformulieren
Stattdessen handelt es sich um eine Hybrid Bridged App, da die nativen \ac{UI}-Komponenten über eine Bridge mit dem JavaScript Code kommunizieren.