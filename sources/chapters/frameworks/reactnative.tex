\section{ReactNative}
\label{sec:Frameorks_ReactNative}

Das von Meta, ehemals Facebook, entwickelte ReactNative Framework ermöglicht die Cross-Plattform Entwicklung mit Webtechnologien.
Dabei kommt die verbreitete \ac{UI}-Bibliothek React zum Einsatz, um die Benutzeroberfläche zu erstellen.
Als Zielplattformen werden neben Android und iOS unter anderem Windows und macOS unterstützt \cite{ReactNative}.
Ein großer Vorteil des Frameworks ist die Verwendung von React für die Umsetzung der \ac{UI}.
Da viele Webentwickler mit React vertraut sind, fällt die Einarbeitung in ReactNative leicht, was zu einer großen Verbreitung des Frameworks führt \cite{Appfigures_TopSDKs,Stackoverflow_2022}.
Außerdem lässt sich ReactNative vergleichsweise einfach in bestehende native Projekte integrieren, was die inkrementelle Migration von Projekten ermöglicht.
Diese Integrationsmöglichkeit führt auch dazu, dass neben den Apps von Meta, weitere populäre Apps wie die Microsoft Office Apps, Pinterest und WordPress teile ihrer Funktionalität mit ReactNative umsetzen \cite{ReactNative_Showcase}.


Obwohl ReactNative wie Cordova und Capacitor auf Webtechnologien, insbesondere JavaScript, basiert, wählt das Framework einen anderen Ansatz, um die Cross-Plattform Entwicklung zu ermöglichen.
Mit Ionic können Cordova und Capacitor Apps ebenfalls React als \ac{UI}-Bibliothek verwenden.
Dabei ist die Oberfläche komplett als Webanwendung implementiert und wird in einer WebView der jeweiligen Plattform dargestellt.
ReactNative verzichtet hingegen auf den Einsatz einer WebView und nutzt stattdessen die nativen \ac{UI}-Komponenten der jeweiligen Plattformen.
Der JavaScript Code wird nicht, wie bei Foreign Language Apps nach Nunkesser \cite{Nunkesser_Taxonomy_Apps}, in nativ ausführbaren Code übersetzt, sondern in einer JavaScript-Runtime ausgeführt.
Somit lässt sich ReactNative weder komplett den Hybrid Web Apps noch den Foreign Language Apps zuordnen.
Stattdessen handelt es sich um Hybrid Bridged Apps, da Webtechnologien und native \ac{UI}-Elemente in einer App kombiniert werden.


Aktuell wird die Architektur von ReactNative auf einen neuen Ansatz umgestellt, der Probleme der bisherigen Architektur beheben soll.
Da jedoch noch beide Ansätze unterstützt werden, wird hier sowohl auf die bisherige als auch auf die neue Architektur eingegangen.

Die bisherige Architektur setzt sich aus zwei Komponenten zusammen, welche über die sogenannte ReactNative Bridge miteinander kommunizieren.
Die Bridge trennt dabei den JavaScript Code von der nativen Umgebung.
Der JavaScript-Code wird in einem eigenen Thread innerhalb der JavaScriptCore Ausführungsumgebung, einem Teil der Open-Source Browser Engine WebKit, ausgeführt.
Dieser Thread wird auch als JavaScript-Thread oder kurze JS-Thread bezeichnet.
Unter iOS ist die WebKit Engine die einzige erlaubte Browser Engine und dementsprechend standardmäßig auf allen iOS Geräten installiert und kann von ReactNative-Apps genutzt werden.
Android liefert die JavaScriptCore nicht mit, weshalb ReactNative-Apps für Android JavaScriptCore mitliefern müssen und demnach größer ausfallen \cite{Dragomir_ReactNative,Nawrocki_Comparison_Hybrid_Native_Frameworks}.
Die \ac{UI} wird von der nativen Umgebung, dem sogenannten nativen Thread, gerendert.
Die Bridge übernimmt im Wesentlichen die Kommunikation zwischen den beiden Thread.
Render-Anweisungen werden vom JS-Thread an den nativen Thread gesendet und umgekehrt werden Eingaben und Events vom nativen Thread an den JS-Thread übermittelt.
Die Kommunikation erfolgt über Nachrichten, welche im \ac{JSON} Format serialisiert übertragen werden \cite{Dragomir_ReactNative}.

% TODO: Performance problem dieser architektur und Weg zur neuen -> \cite{Cook_ReactNativeBridge}



%TODO: Architektur


%TODO: Beliebtheit, mögliche Performanceprobleme