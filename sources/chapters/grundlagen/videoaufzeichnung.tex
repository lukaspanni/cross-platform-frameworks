\section{Use Case: Videoaufzeichnung}
\label{sec:Videoaufzeichnung}

Zunächst sollen im Folgenden die Grundlagen der digitalen Videoaufzeichnung und die dabei wichtigen Parameter erläutert werden.
Eine digitale Videoaufzeichnung besteht im Wesentlichen aus einer Folge von Bildern und eventuell einer Tonspur.
In dieser Arbeit liegt der Fokus auf der visuellen Komponente der Videoaufzeichnung.
Digitalkameras, wie sie auch in Smartphones zu finden sind, verwenden zur Aufnahme der einzelnen Bilder einen oder mehrere \ac{CCD} oder \ac{CMOS} Sensoren.
Diese Sensoren verwenden den Photoeffekt, um einfallende Lichtintensität in elektronische Signale umzuwandeln.
Jeder Sensor besteht aus einem Raster kleiner, als Pixel bezeichnete, Sensorelemente.
Die Lichtintensität an jedem einzelnen Pixel wird zu einem Digitalbild kombiniert \cite[S. 63ff.]{Szeliski_ComputerVision}.
Um Farbbilder zu gewinnen wird heute meist ein Sensor mit einer Bayer-Matrix, einem Muster aus Farbfiltern, verwendet.
So ist jeder Pixel nur für jeweils eine der Grundfarben Rot, Grün oder Blau zuständig.
Die Farbwerte für jedes Pixel werden dann durch Berechnung anhand der umliegenden Pixel ermittelt \cite[S. 420ff.]{Schmidt_Videotechnik}.

Das von einer digitalen Kamera aufgenommene Bild ist insbesondere von der Lichtintensität abhängig.
Diese wird vor allem durch die Belichtungszeit und die Größe der Blendenöffnung bestimmt.
Außerdem kann das Bild durch Variation der Sensorempfindlichkeit beeinflusst werden.
Die Belichtungszeit ist die Zeit, die der Sensor dem einfallenden Licht ausgesetzt ist \cite[S. 390ff.]{Schmidt_Videotechnik}.
Sensorempfindlichkeit bestimmt insbesondere die Verstärkung der elektronischen Signale und wird heute meist als \acs{ISO}-Wert, angelehnt an den zugrundeliegenden Standard der \acf{ISO}, angegeben \cite[S. 412ff.]{Schmidt_Videotechnik}.
Die Blendenöffnung ist der Durchmesser der variablen Blende, welche die Öffnung bestimmt, durch die das Licht auf den Sensor fällt \cite[S. 444ff.]{Schmidt_Videotechnik}.
Die Parameter Blendenöffnung, Belichtungszeit und Sensorempfindlichkeit werden auch als Belichtungsparameter bezeichnet.

Neben den Parametern für die Belichtung wird das Bild auch durch die Brennweite des verwendeten Objektivs und den Fokus bestimmt.
Durch Veränderung der Brennweite lässt sich der Bildausschnitt verändern und eine Vergrößerung beziehungsweise Verkleinerung von Objekten erreichen.
Der Fokus bestimmt, welche Entfernung des Bildes scharf dargestellt wird und wird in Amateurkameras häufig automatisch eingestellt \cite[S. 445ff.]{Schmidt_Videotechnik}.



% Zusätzliche Parameter (die man im (semi-) professionellen Bereich beeinflussen will): Weißabgleich, Fokus, Format
% Zusätzlich für Video: Bildrate
% Eventuell auh Speicherformat

% Welche Parameter sind prinzipiell bei Smartphones anpassbar, welche nicht