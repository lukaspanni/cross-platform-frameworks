\chapter{Fazit}
\label{ch:fazit}


Cross-Plattform Frameworks können die Entwicklung von Apps für die beiden mobilen Plattformen Android und iOS deutlich erleichtern.
Während der klassische Ansatz die Entwicklung mit der, vom Betriebssystem vorgegebenen Programmiersprache, vorsieht, ermöglichen Cross-Plattform Frameworks die überwiegende Verwendung einer gemeinsamen Programmiersprache.
So lässt sich ein großer Anteil des Codes für beide Plattformen wiederverwenden, was die Entwicklungszeit und Entwicklungskosten deutlich reduziert.
Allerdings müssen Apps für viele Funktionen auf die \acp{API} des Betriebssystems zurückgreifen.
Diese Zugriffe müssen plattformspezifisch umgesetzt werden.
Für häufig verwendete Funktionen existieren für die meisten Frameworks entsprechende Plugins, sodass auch hier die Wiederverwendung von Code möglich ist.
Jedoch ist ein Plugin immer für einen bestimmten Anwendungsfall konzipiert und stellt eventuell nicht alle Funktionen zur Verfügung, die für andere Anwendungsfälle benötigt werden.
Im Rahmen der vorliegenden Arbeit wurde untersucht, inwiefern die Entwicklung einer App zur Videoaufzeichnung mit den aktuell populärsten Cross-Plattform Frameworks umgesetzt werden kann.
Durch die verschiedenen Ansätze der Frameworks und die unterschiedliche Verfügbarkeit von Plugins gibt es große Unterschiede in der Einsetzbarkeit.


Die Frameworks Flutter, Ionic beziehungsweise Ionic mit Cordova, Xamarin und React Native wurden als populärste Cross-Plattform Frameworks identifiziert.
In der genauen Betrachtung der Funktionsweise in \autoref{ch:frameworks} wird klar, dass die Frameworks unterschiedliche Ansätze bei der Umsetzung von plattformübergreifender Funktionalität verfolgen.
Unterschiede bestehen nicht nur in der verwendeten Programmiersprache, sondern auch in der grundlegenden Architektur und der Art des Zugriffs auf native Funktionen.
Alle Frameworks ermöglichen jedoch prinzipiell den Zugriff auf alle \acp{API} der mobilen Betriebssysteme.
Nur mit Xamarin ist eine direkte Nutzung der Schnittstellen möglich, ohne für die jeweilige Plattform nativen Code entwickeln zu müssen.
Für die anderen Frameworks steht eine Vielzahl von Plugins zur Verfügung, sodass viele Anwendungsfälle umsetzbar sein sollten.
Damit ist es mit jedem der untersuchten Frameworks grundsätzlich möglich, eine App zur Videoaufzeichnung zu entwickeln.


Für die Implementierung von prototypischen Apps wurde jeweils nur auf verfügbare Plugins oder Bibliotheken zurückgegriffen.
Keine Funktion wurde direkt in nativen Code umgesetzt.
Die essenziellen funktionalen Anforderungen, definiert in \autoref{ch:bewertungskriterien}, konnten mit allen Frameworks umgesetzt werden.
Bei der Umsetzung der optionalen Anforderungen und der Untersuchung von Startzeit, App-Größe und Performance in einem Ende-zu-Ende-Test, wurden jedoch deutliche Unterschiede zwischen den Frameworks festgestellt.

React Native hat sich als am besten geeignetes Framework für die Entwicklung einer App zur Videoaufzeichnung herausgestellt.
Mit keinem anderen Framework konnten mehr der Parameter für die Videoaufzeichnung eingestellt werden.
Dennoch sind nicht alle wünschenswerten Parameter verfügbar.
Außerdem ist der React-Native Prototyp im Ende-zu-Ende-Test aufgrund sehr effizientem Zugriff auf native Funktionen am performantesten.
Einziges Problem ist die sehr große App-Größe unter iOS, welche jedoch vermutlich auf einen Fehler im Zusammenspiel mit der JavaScript Engine Hermes zurückzuführen ist.
Einzig Flutter wird als weiters Framework für die Entwicklung einer App zur Videoaufzeichnung empfohlen.
In vielen Aspekten sind die Ergebnisse von Flutter und React Native vergleichbar.
Die Einstellbarkeit der Parameter für die Videoaufzeichnung ist jedoch bei dem für Flutter verwendeten Plugin nicht so umfangreich wie bei React Native.
Die Verfügbarkeit geeigneter Plugins konnte allgemein als entscheidendes Kriterium für die Eignung der Frameworks identifiziert werden.
Da für Cordova und Xamarin keine Plugins existieren, welche den Anwendungsfall der Videoaufzeichnung gut unterstützen, werden diese Frameworks nicht empfohlen.
Technisch ist die Umsetzung der Funktionalität in Form von Plugins für diese Frameworks möglich.
Als Grund für das Fehlen von Plugins wurde die im Vergleich geringere Popularität und Verbreitung von Cordova und Xamarin identifiziert.


Abschließend kann festgehalten werden, dass zur Entwicklung einer App zur Videoaufzeichnung React Native und Flutter in Betracht gezogen werden können.
Jedoch lassen sich mit keinem der verfügbaren Plugins für die Frameworks, alle wünschenswerten Parameter für die Videoaufzeichnung einstellen.
Werden alle Parameter benötigt, müssen entweder die bestehenden Plugins erweitert werden, oder die Entwicklung nativ erfolgen.
Sofern nicht alle der Parameter benötigt werden, kann die Nutzung der Cross-Plattform Frameworks allerdings eine deutliche Zeit- und Kostenersparnis bringen.
