\chapter{Bewertungskriterien}
\label{ch:Bewertungskriterien}

Bevor die Evaluation der Frameworks für die Umsetzung des Anwendungsfalls der Videoaufzeichnung in \autoref{ch:Evaluation} erfolgen kann, müssen zunächst die Anforderungen an die prototypische Anwendung sowie die Kriterien für die Bewertung der Frameworks definiert werden. 
Außerdem muss definiert werden, wie die Implementierung erfolgen soll.

\section{Anforderungen an den Prototypen und die Implementierung}
\label{sec:anforderungen}
Um den Use Case der Videoaufzeichnung bewerten zu können, müssen folgende Anforderungen erfüllt werden:
\begin{itemize}
  \item Der Prototyp muss die Aufzeichnung von Video (mit Audio) ermöglichen.
  \item Der Prototyp muss die Speicherung von Videoaufzeichnungen im lokalen, für den Nutzer zugreifbaren Speicher, ermöglichen.
  \item Der Prototyp soll die Wiedergabe von aufgezeichnetem Video ermöglichen.
  \item Der Prototyp soll die Einstellung der in \autoref{tab:parameter_support} als einstellbar eingestuften Parameter ermöglichen.
\end{itemize}

Können mit „muss“ gekennzeichnete Anforderungen mit einem bestimmten Framework nicht erfüllt werden, so ist das Framework als nicht geeignet für die Umsetzung des Anwendungsfalls zu bewerten.
Die Anforderungen, die mit „soll“ gekennzeichnet sind, sind für die Basisfunktionalität nicht zwingend erforderlich, für einen sinnvollen Einsatz des Frameworks für die Videoaufzeichnung jedoch wünschenswert.

Bei der Implementierung sollen nur die Abstraktionen der jeweiligen Frameworks oder verfügbare Open-Source Komponenten verwendet werden.
Insbesondere soll keine Funktionalität mit denen für die Plattformen nativen Programmiersprachen umgesetzt werden.
Dies würde den Vorteil einer Plattformübergreifenden Entwicklung aufheben und die Vergleichbarkeit der Frameworks einschränken.
In diesem Fall könnte nur der Mechanismus zum Aufruf von nativem Code der jeweiligen Frameworks verglichen werden, was nicht Fokus der vorliegenden Arbeit ist.
Weiterhin soll im Code für die Videoaufzeichnung möglichst keine Unterscheidung zwischen den Plattformen Android und iOS notwendig sein.


\section{Bewertungskriterien}

Zur Bewertung des Prototyps soll zunächst die Umsetzung der Anforderungen aus \autoref{sec:anforderungen} herangezogen werden.
Dabei muss zumindest die Grundfunktionalität der Videoaufzeichnung erfüllt werden.
Bei den einstellbaren Parametern ist der größte Unterschied zwischen den Frameworks zu erwarten, weshalb die Einstellbarkeit besonders betrachtet werden soll.
In diesem Zusammenhang soll nicht nur untersucht werden, welche Parameter sich einstellen lassen, sondern auch die erlaubten Wertebereiche und mögliche Einschränkungen.


Zudem sollen allgemeine Kriterien bewertet werden.
Dazu werden die Größe und Startzeit der Anwendung und die Prozessor- und Speicherauslastung während der Aufzeichnung betrachtet.
Darüber hinaus soll die Zeit gemessen werden, die benötigt wird, ein Video von zehn Sekunden Länge aufzuzeichnen und abzuspeichern.
Diese Kriterien entsprechen im Wesentlichen den Kriterien, welche von Nawrocki \textit{et al.} \cite{Nawrocki_Comparison_Hybrid_Native_Frameworks} für die Bewertung von React Native, Xamarin und Flutter im Kontext zweier generischer Testanwendungen verwendet werden.
Auch Bi{\o}rn-Hansen \textit{et al.} \cite{Biorn-Hansen_PerformanceOverhead_CrossPlatform} verwenden vorrangig diese Kriterien.

Durch die unterschiedliche Funktionsweise der Frameworks, sind bei der Größe der Anwendung größere Unterschiede zu erwarten.
Bewertet werden soll dabei jeweils eine für die Auslieferung optimiert Version der Anwendung in den Formaten \ac{IPA} für iOS und \ac{AAB} für Android.

Die Startzeit der Anwendung ist insbesondere für die Nutzbarkeit der Anwendung von Bedeutung und soll aus einem Mittelwert mehrerer Messungen ermittelt werden.
Dazu werden jeweils die vom Betriebssystem bereitgestellten Funktionen zum kompletten Beenden der Anwendung verwendet, sodass Kaltstartzeiten gemessen werden können.
Unter Android wird zur Messung der Zeiten auf die vom Betriebssystem bereitgestellten Metriken \ac{TTID} und \ac{TTFD} zurückgegriffen.
Da unter iOS keine vergleichbaren Metriken bereitgestellt werden, muss die Zeit zwischen klicken auf das App-Symbol und der Anzeige der kompletten Anwendung gemessen werden.
Dazu wird der Bildschirm des Testgeräts mit einer Bildrate von 120 Bildern pro Sekunde aufgezeichnet und anhand der Aufnahmen die Zeit berechnet.

Prozessor- und Speicherauslastung während einer Videoaufzeichnung sollen eventuelle Abweichungen zwischen den Frameworks im Hinblick auf Auslastung der verfügbaren Ressourcen aufzeigen.
Dazu wird die Auslastung der CPU und des Arbeitsspeichers mit den Tools der Betriebssystemhersteller gemessen.
