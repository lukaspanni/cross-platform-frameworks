\chapter{Auswahl eines Cross-Plattform Frameworks}
\label{ch:auswahl}

In diesem Kapitel soll auf Basis der Erkenntnisse eine Empfehlung zur Auswahl eines Cross-Plattform Frameworks gegeben werden.
Grundsätzlich ist die Auswahl eines Frameworks vom konkreten Anwendungsfall abhängig.
Wie bereits gezeigt, ist React Native für die Videoaufzeichnung von allen untersuchten Frameworks am besten geeignet.
Außerdem lässt sich diese Anwendung auch mit Flutter gut umsetzen.
Bei der Videoaufzeichnung handelt es sich um einen Anwendungsfall mit vergleichsweise hohen Anforderungen an den Zugriff auf Funktionen des Betriebssystems.
Zusätzlich zu den Funktionen des Kamerasystems ist ein Zugriff auf das Dateisystem der Plattform notwendig.
Der Zugriff auf native Funktionen ist bei React Native effizienter umgesetzt als bei Flutter oder Cordova.
Grundlage dafür ist die neue Architektur und der damit einhergehende Verzicht auf Nachrichtenbasierte Kommunikation, wie in \autoref{sec:frameorks_reactnative} bereits erläutert.
Plugins können somit effizienter auf native Funktionen zugreifen, was sich positiv auf die Performance auswirkt.
Die Ergebnisse des Ende-zu-Ende-Tests in \autoref{sec:evaluation_allgemein} zeigen die höhere Performance von React Native im Vergleich zu Flutter.

Xamarin stellt einen ähnlich effizienten Zugriff auf native Funktionen bereit und erlaubt darüber hinaus auch die Verwendung nativer \acp{API} direkt in C\#.
Allerdings sind deshalb kaum Plugins vorhanden, weshalb mehr Code selbst entwickelt werden muss.
Außerdem ist die allgemeine Performance der Mono-Ausführungsumgebung für C\# schlechter als die von React Native eingesetzten JavaScript Ausführungsumgebungen.
Deshalb wird React Native für Anwendungsfälle mit vielen Zugriffen auf native \acp{API} empfohlen, solange die Zugriffe über Plugins abgedeckt sind.
Dies ist bei gängigen Anwendungsfällen zu erwarten.
Werden jedoch sehr spezielle \acp{API} benötigt, ist Xamarin trotz schlechterer Performance die bessere Wahl, da der Aufwand der Entwicklung eigener entfällt.
Allerdings wird dabei auch der größte Vorteil der Cross-Plattform Entwicklung abgeschwächt, sodass in diesem Fall die native Entwicklung in Betracht gezogen werden kann.
Insbesondere, wenn ein großer Teil der Anwendung aus nativen Zugriffen besteht, ist native Entwicklung vorzuziehen.

Bei Anwendungsfällen mit wenig Zugriffen auf native \acp{API} und damit geringerem Bedarf für Plugins sind die Unterschiede zwischen den Frameworks kleiner.
Apps, welche im wesentlichen Daten aus dem Internet laden und anzeigen, sind uneingeschränkt mit jedem Framework umsetzbar.
Zum Beispiel Nachrichten-Apps oder einfache Apps für Online-Shopping benötigen nicht zwingend Zugriff auf native \acp{API}, um funktionsfähig zu sein.

Die untersuchten allgemeinen Kriterien, wie Startzeit und App-Größe, können herangezogen werden, um in so einem Fall eine Empfehlung aussprechen zu können.
Die Startzeit kann vor allem die Nutzbarkeit einer App und den subjektiven Eindruck der Performance beeinflussen.
App-Größe hingegen bestimmt die Downloadzeit und damit die Wartezeit bis zur ersten Nutzung der App und den Speicherbedarf auf dem Smartphone.
Bei weiter steigenden Internetgeschwindigkeiten und immer mehr Speicherplatz auf Smartphones verliert die App-Größe jedoch zunehmend an Bedeutung.
Die geringste Größe im Durchschnitt der Plattformen wird von Ionic mit Cordova erreicht, gefolgt von Flutter, Xamarin und React Native.
Flutter erzielt die im Durchschnitt geringste Startzeit, wobei React Native sehr ähnliche Zeiten erreicht.
Xamarin und Ionic mit Cordova sind hingegen deutlich langsamer.

Werden nur App-Größe und Startzeit berücksichtigt, ist Flutter damit für Apps mit wenigen nativen Aufrufen zu empfehlen.
Außerdem stehen für Flutter, wie für React Native, viele Plugins zur Verfügung, sodass auch bei Bedarf auf nativen Code zurückgegriffen werden kann.