\chapter{Populäre Cross-Plattform Frameworks}
\label{ch:Frameworks}

Im Folgenden sollen die vier populärsten Frameworks für die Cross-Plattform Entwicklung näher betrachtet werden.
Dabei liegt der Fokus auf der Funktionsweise der Frameworks und der Einordnung in eine der Kategorien nach Nunkesser, welche in \autoref{sec:Entwicklungsansaetze} vorgestellt wurden.
Bevor auf einzelne Frameworks eingegangen werden kann, müssen zunächst die aktuell populärsten Frameworks identifiziert werden.
Dazu werden insbesondere die StackOverflow Developer Surveys \cite{Stackoverflow_2021} \cite{Stackoverflow_2022} der letzten zwei Jahre herangezogen, welche Entwickler unter anderem zu ihren bevorzugten und im professionellen Bereich eingesetzten Frameworks befragen.
Diese Frameworks müssen im nächsten Schritt auf ihre Cross-Plattformfähigkeit hin untersucht werden.
Weiterhin wird auf die Umfrage \cite{Statista_UsedCrossPlatformFrameworks} zurückgegriffen, welche die am häufigsten eingesetzten Cross-Plattfom Frameworks zwischen 2019 und 2021 ermittelt hat.
In dieser Umfrage gab etwa ein Drittel der befragten Entwickler im Bereich der mobilen App-Entwicklung an, ein Cross-Plattform Framework zu verwenden.
Im Jahr 2021 verwendeten 42 \% von diesen Entwicklern Flutter, was es zum am häufigsten genutzten Framework macht.
ReactNative kommt in dieser Umfrage auf 38 \%, Cordova auf 16 \% und Ionic ebenfalls auf 16 \% \cite{Statista_UsedCrossPlatformFrameworks}.
Das Problem hierbei ist, dass Ionic kein Cross-Plattform Framework ist, sondern selbst auf die Cross-Plattform Frameworks Cordova oder Capacitor aufsetzt \cite{Ionic_Docs}.
Deshalb müssen Ionic und Cordova in dieser Umfrage zusammengefasst werden und werden gemeinsam betrachtet.
Das demnach am vierthäufigsten eingesetzte Framework ist Xamarin mit 11 \% \cite{Statista_UsedCrossPlatformFrameworks}.
In der StackOverflow Developer Survey von 2021 \cite{Stackoverflow_2021} sind Cordova und Xamarin nicht aufgeführt.
Da diese Umfrage sich nicht nur auf Entwickler mobiler Anwendungen fokussiert, sind hier die Anteile der eingesetzten Frameworks deutlich kleiner.
Zur Bestimmung der populärsten Frameworks sind diese Daten dennoch verwendbar.
Die einzigen vorkommenden Cross-Plattform Frameworks sind ReactNative (16,48 \%), Flutter (13,35 \%) und Cordova (8,67 \%), welche damit auch die populärsten sind.
2022 sind in der StackOverflow Developer Survey \cite{Stackoverflow_2022} Flutter (13,62 \%), ReactNative (12,56 \%), Ionic (5,97 \%) und Xamarin (5,54 \%), die vier am häufigsten eingesetzten Frameworks.
Damit kommen alle herangezogenen Umfragen auf die gleiche Menge der populärsten Cross-Plattform Frameworks.
Die genaue Rangfolge ist in dieser ARbeit nicht relevant, aber auch hier stimmen die Umfragen im Wesentlichen überein.

In den folgenden Abschnitten werden die vier Frameworks Flutter, ReactNative, Ionic (mit Cordova) und Xamarin näher betrachtet.

