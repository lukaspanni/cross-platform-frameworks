\chapter{Einleitung}
\label{ch:Einleitung}

% TODO: Relevanz von Video anhand der populärsten Apps.

\section{Motivation}
\label{sec:Motivation}

Auf dem Markt mobiler Betriebssysteme dominieren aktuell die beiden Systeme Android und iOS.
Alle alternativen Betriebssysteme für mobile Geräte zusammen erreichen im September 2022 weltweit weniger als 1 \% Marktanteil \cite{mobile_market_share}.
Die Entwicklung von Anwendungen für Mobilgeräte, bezeichnet als Apps, muss sich dementsprechend vorrangig auf diese beiden Systeme konzentrieren.
Um möglichst viele Nutzer zu erreichen, sollten mobile Anwendungen für beide Betriebssysteme gleichermaßen bereitgestellt werden.
Für die Programmierung von Anwendungen sehen allerdings beide Systeme eine andere Programmiersprache und die Verwendung spezieller \acp{API} vor.
In der Folge müssen zwei eigenständige, sogenannte native Apps entwickelt werden.
Komponenten oder Klassen von einer Plattform wiederzuverwenden ist durch die Unterschiede in den verwendeten Programmiersprachen nicht möglich.

In den letzten Jahren haben sich zusätzlich zum nativen Ansatz verschiedene weitere Ansätze etabliert, mobile Anwendungen sowohl für Android als auch iOS zu entwickeln.
Diese Ansätze haben jeweils das Ziel, die Wiederverwendung von Komponenten und Code über die Plattformgrenzen hinweg zu fördern.
Die Art und Weise, wie dieses Ziel erreicht werden soll unterscheidet sich jedoch zwischen den Ansätzen und zwischen verschiedenen konkreten Frameworks und Tools.

In ... werden die verschiedenen grundsätzlichen Ansätze der Entwicklung mobiler Anwendungen anhand der Taxonomie von Nunkesser ... vorgestellt.
Ausgewählte Frameworks für die Plattformübergreifende Entwicklung werden in ... näher erklärt und in ... im Rahmen einer Beispielanwendung für Videoaufnahmen ausführlich untersucht.
